% UML Class Diagram - Tournament Management System (Gestion Tournoi)
% This LaTeX file contains a comprehensive UML class diagram showing the core architecture
% Usage: Include in your main LaTeX document with: % UML Class Diagram - Tournament Management System (Gestion Tournoi)
% This LaTeX file contains a comprehensive UML class diagram showing the core architecture
% Usage: Include in your main LaTeX document with: % UML Class Diagram - Tournament Management System (Gestion Tournoi)
% This LaTeX file contains a comprehensive UML class diagram showing the core architecture
% Usage: Include in your main LaTeX document with: % UML Class Diagram - Tournament Management System (Gestion Tournoi)
% This LaTeX file contains a comprehensive UML class diagram showing the core architecture
% Usage: Include in your main LaTeX document with: \input{UML_DiagrammeClasses}

\documentclass[12pt]{article}
\usepackage[utf-8]{inputenc}
\usepackage[french]{babel}
\usepackage{tikz}
\usepackage{tikz-uml}
\usepackage{geometry}
\geometry{landscape, margin=0.5cm}

\title{Diagramme de Classes UML - Système de Gestion de Tournois}
\author{GETP2 - Yong Ngantcha Salzenstein}
\date{\today}

\begin{document}

\begin{center}
\begin{tikzpicture}[scale=0.9, every node/.style={anchor=west}]

% ============================================================================
% BACKEND - MODEL LAYER (Métier / Persistence)
% ============================================================================

% CLASS: Utilisateur
\begin{class}[text width=4cm]{Utilisateur}{-8, 14}
    \attribute{- id: int}
    \attribute{- surnom: String}
    \attribute{- pass: String}
    \attribute{- isAdmin: boolean}
    \operation{+ getId(): int}
    \operation{+ getSurnom(): String}
    \operation{+ isAdmin(): boolean}
    \operation{+ getRole(): int}
\end{class}

% CLASS: UserSession (Singleton-like)
\begin{class}[text width=5cm]{UserSession}{-8, 8}
    \attribute{- static curUser: Utilisateur}
    \attribute{- static currentTournoiId: Integer}
    \attribute{- static currentTournoiName: String}
    \operation{+ static login(Utilisateur): void}
    \operation{+ static logout(): void}
    \operation{+ static userConnected(): boolean}
    \operation{+ static getCurrentTournoiId(): Optional<Integer>}
    \operation{+ static setCurrentTournoi(id, name): void}
\end{class}

% CLASS: Parametre (Tournament Settings)
\begin{class}[text width=4.5cm]{Parametre}{3, 14}
    \attribute{- id: int}
    \attribute{- nomTournoi: String}
    \attribute{- sport: String}
    \attribute{- nombreTerrains: int}
    \attribute{- nombreJoueursParEquipe: int}
    \operation{+ getId(): int}
    \operation{+ getNomTournoi(): String}
    \operation{+ getSport(): String}
    \operation{+ saveInDB(Connection): int}
\end{class}

% CLASS: ClasseMiroir (Abstract Base for DB Entities)
\begin{class}[text width=5cm]{«abstract» ClasseMiroir}{3, 8}
    \attribute{- id: int}
    \operation{+ getId(): int}
    \operation{+ setId(int): void}
    \operation{# abstract saveSansId(Connection): Statement}
    \operation{+ saveInDB(Connection): int}
    \operation{+ loadFromDB(Connection, int): void}
\end{class}

% CLASS: Joueur
\begin{class}[text width=4.5cm]{Joueur}{-2, 3}
    \attribute{- prenom: String}
    \attribute{- nom: String}
    \attribute{- age: int}
    \attribute{- sexe: String}
    \attribute{- taille: double}
    \operation{+ getPrenom(): String}
    \operation{+ getNom(): String}
    \operation{+ getAge(): int}
\end{class}

% CLASS: Equipe
\begin{class}[text width=4.5cm]{Equipe}{3, 3}
    \attribute{- nomEquipe: String}
    \attribute{- dateCreation: LocalDate}
    \attribute{- joueurs: List<Joueur>}
    \operation{+ getNomEquipe(): String}
    \operation{+ getJoueurs(): List<Joueur>}
    \operation{+ addJoueur(Joueur): void}
\end{class}

% CLASS: Terrain
\begin{class}[text width=4.5cm]{Terrain}{8, 3}
    \attribute{- nomTerrain: String}
    \attribute{- numero: int}
    \operation{+ getNomTerrain(): String}
    \operation{+ getNumero(): int}
    \operation{+ toString(): String}
\end{class}

% CLASS: GestionMatchs (Service Layer)
\begin{class}[text width=5.5cm]{GestionMatchs}{-8, 1}
    \attribute{- static MatchRow: record}
    \attribute{- static ButeurRow: record}
    \operation{+ static creerRencontres(): void}
    \operation{+ static calculerScores(): void}
    \operation{+ static getClassement(): List<ClassementRow>}
    \operation{+ static enregistrerBut(): void}
\end{class}

% CLASS: ConnectionPool (Database Access)
\begin{class}[text width=5cm]{ConnectionPool}{3, -2}
    \attribute{- static dataSource: HikariDataSource}
    \attribute{- static dbUrl: String}
    \operation{+ static init(): void}
    \operation{+ static getConnection(): Connection}
    \operation{+ static close(): void}
\end{class}

% ============================================================================
% FRONTEND LAYER (Vaadin Views)
% ============================================================================

% CLASS: BaseLayout (Navigation Container)
\begin{class}[text width=5cm]{BaseLayout}{-8, -4}
    \attribute{- gridCurrentTournoi: Grid}
    \attribute{- contentLayout: VerticalLayout}
    \operation{- createHeader(): void}
    \operation{- createDrawer(): void}
    \operation{- logout(): void}
    \operation{- switchTournament(): void}
\end{class}

% CLASS: VuePrincipale (Home)
\begin{class}[text width=4.5cm]{VuePrincipale}{-2, -4}
    \attribute{- container: VerticalLayout}
    \operation{+ construireInterfaceConnectee(): void}
    \operation{+ construireInterfaceNonConnectee(): void}
\end{class}

% CLASS: VueConnexion (Login/SignUp)
\begin{class}[text width=4.5cm]{VueConnexion}{3, -4}
    \attribute{- tfUsername: TextField}
    \attribute{- pfPassword: PasswordField}
    \operation{- gererConnexion(): void}
    \operation{- gererInscription(): void}
\end{class}

% CLASS: VueJoueur (Player Management)
\begin{class}[text width=4.5cm]{VueJoueur}{8, -4}
    \attribute{- grid: Grid<Joueur>}
    \attribute{- dialogForm: Dialog}
    \operation{- chargerJoueurs(): void}
    \operation{- ajouterJoueur(): void}
    \operation{- supprimerJoueur(): void}
\end{class}

% CLASS: VueEquipe (Team Management)
\begin{class}[text width=4.5cm]{VueEquipe}{-2, -8}
    \attribute{- grid: Grid<Equipe>}
    \attribute{- joueursList: List<Joueur>}
    \operation{- chargerEquipes(): void}
    \operation{- associerJoueurs(): void}
\end{class}

% CLASS: VueMatch (Match Management)
\begin{class}[text width=4.5cm]{VueMatch}{3, -8}
    \attribute{- gridRencontres: Grid}
    \attribute{- gridScores: Grid}
    \operation{- chargerMatchs(): void}
    \operation{- enregistrerScore(): void}
\end{class}

% CLASS: VueBut_alle (Scoring Board)
\begin{class}[text width=5cm]{VueBut_alle}{8, -8}
    \attribute{- gridCurrentTournoi: Grid}
    \attribute{- gridAllTournois: Grid}
    \operation{- loadButeurs(): void}
    \operation{- buildPodium(): HorizontalLayout}
\end{class}

% ============================================================================
% DATABASE LAYER (Conceptual)
% ============================================================================

% DATABASE ICON
\node[draw, rectangle, fill=cyan!20, text width=8cm, minimum height=1.5cm] at (3, -11.5) {
    \textbf{Base de Données Relationnelle} \\
    Tables: \texttt{utilisateur, parametre, joueur, equipe, terrain, rencontre, but}
};

% ============================================================================
% RELATIONSHIPS / ASSOCIATIONS
% ============================================================================

% UserSession uses Utilisateur
\draw [->, dashed, line width=1.5pt] (-2, 8) -- (1, 14);
\node[text, font=\tiny] at (0, 11) {\textbf{0..1}};

% Parametre inherits from ClasseMiroir
\draw [->, line width=1.5pt] (5.5, 8) -- (6, 12);
\node[text, font=\tiny] at (6.5, 10) {\textbf{inherits}};

% Joueur inherits from ClasseMiroir
\draw [->, line width=1.5pt] (2.5, 8) -- (0.5, 3);
\node[text, font=\tiny] at (1, 5.5) {\textbf{inherits}};

% Equipe inherits from ClasseMiroir
\draw [->, line width=1.5pt] (4.5, 8) -- (5.5, 3);
\node[text, font=\tiny] at (5.5, 5.5) {\textbf{inherits}};

% Terrain inherits from ClasseMiroir
\draw [->, line width=1.5pt] (6.5, 8) -- (10, 3);
\node[text, font=\tiny] at (8.5, 5.5) {\textbf{inherits}};

% Equipe contains Joueurs
\draw [->, line width=1.5pt] (3, 3) -- (2, 3);
\node[text, font=\tiny] at (2.5, 2.5) {\textbf{0..*}};
\node[text, font=\tiny] at (2.5, 3.3) {\textbf{contient}};

% GestionMatchs uses Joueur, Equipe, Terrain
\draw [->, dashed, line width=1pt] (-5, 1) -- (-1, 3);
\draw [->, dashed, line width=1pt] (-4, 1) -- (2, 3);
\draw [->, dashed, line width=1pt] (-3, 1) -- (8, 3);

% BaseLayout extends AppLayout (Vaadin)
\node[text, font=\tiny] at (-6, -4.5) {\textbf{extends AppLayout}};

% VuePrincipale extends BaseLayout
\draw [->, line width=1.5pt] (-2, -4) -- (-6.5, -4);
\node[text, font=\tiny] at (-4, -3.7) {\textbf{extends}};

% VueConnexion uses Utilisateur
\draw [->, dashed, line width=1pt] (4, -4) -- (0, 14);

% VueJoueur extends BaseLayout
\draw [->, line width=1.5pt] (8.5, -4) -- (-6, -4);
\node[text, font=\tiny] at (2, -3.7) {\textbf{extends}};

% VueJoueur uses Joueur
\draw [->, dashed, line width=1pt] (8.5, -4) -- (1, 3);

% VueEquipe extends BaseLayout
\draw [->, line width=1.5pt] (-2, -8) -- (-6, -4.5);
\node[text, font=\tiny] at (-4, -6) {\textbf{extends}};

% VueEquipe uses Equipe, Joueur
\draw [->, dashed, line width=1pt] (-1, -8) -- (3, 3);
\draw [->, dashed, line width=1pt] (-1.5, -8) -- (-0.5, 3);

% VueMatch extends BaseLayout
\draw [->, line width=1.5pt] (3, -8) -- (-6, -4.5);
\node[text, font=\tiny] at (-1, -6) {\textbf{extends}};

% VueMatch uses GestionMatchs
\draw [->, dashed, line width=1pt] (3, -8) -- (-5, 1);

% VueBut_alle extends BaseLayout
\draw [->, line width=1.5pt] (8.5, -8) -- (-6, -4.5);
\node[text, font=\tiny] at (2, -6) {\textbf{extends}};

% All Views access Database through ConnectionPool
\draw [->, dashed, line width=1pt, color=red] (6, -2) -- (6, -2.5);
\node[text, font=\tiny, color=red] at (6.5, -1.5) {\textbf{SQL}};

% ============================================================================
% LEGEND
% ============================================================================

\node[draw, rectangle, fill=gray!10, text width=18cm, minimum height=2cm] at (6, -13.5) {
    \textbf{Légende:} \\
    \textbf{→} Héritage / Extension \quad \textbf{⇢} Association/Utilisation \quad \textbf{- - →} Dépendance/Accès \\
    \textbf{Couches:} Backend (Model/Métier) = Classes jaunes | Frontend (Vaadin Views) = Classes bleues | DB Accès = Classes vertes
};

\end{tikzpicture}
\end{center}

\newpage

% ============================================================================
% DETAILED DESCRIPTIONS
% ============================================================================

\section*{Description Détaillée de l'Architecture}

\subsection*{1. Couche Métier (Backend Model)}

\begin{description}
    \item[\textbf{Utilisateur}] Représente un utilisateur du système avec ses droits (admin/user). 
        Stocké en base de données avec chiffrement du mot de passe.
    
    \item[\textbf{UserSession}] Gestionnaire de session singleton. Maintient l'état actuel de l'utilisateur 
        connecté et du tournoi sélectionné. Gère la transition login/logout.
    
    \item[\textbf{Parametre}] Définit les paramètres d'un tournoi (nom, sport, nombre de terrains, 
        joueurs par équipe). Hérite de \texttt{ClasseMiroir} pour la persistance.
    
    \item[\textbf{Joueur}] Entité représentant un joueur avec attributs (prénom, nom, âge, taille, sexe).
        Liée au tournoi via \texttt{tournoi\_id}.
    
    \item[\textbf{Equipe}] Groupe de joueurs avec date de création. Contient une collection de \texttt{Joueur}.
    
    \item[\textbf{Terrain}] Ressource physique pour les matchs (numéro, nom).
    
    \item[\textbf{ClasseMiroir}] Classe abstraite de base pour toutes les entités persistantes.
        Gère l'ID auto-généré et les opérations CRUD standard.
    
    \item[\textbf{GestionMatchs}] Service métier pour la création et gestion des rencontres.
        Utilise des DTOs internes (\texttt{MatchRow}, \texttt{ButeurRow}) pour les requêtes complexes.
\end{description}

\subsection*{2. Couche Données (Database Access)}

\begin{description}
    \item[\textbf{ConnectionPool}] Gestionnaire de connexions utilisant HikariCP. 
        Fournit des connexions réutilisables pour optimiser les performances.
    
    \item[\textbf{Schéma Unifié}] Une seule base de données avec colonne \texttt{tournoi\_id} 
        pour isoler les données par tournoi (pas de tables par tournoi).
\end{description}

\subsection*{3. Couche Présentation (Vaadin Frontend)}

\begin{description}
    \item[\textbf{BaseLayout}] Container racine hérité par toutes les vues. Fournit le header 
        (navigation, utilisateur, déconnexion) et le drawer (menu latéral).
    
    \item[\textbf{VuePrincipale}] Accueil avec interface différente selon l'état de connexion.
    
    \item[\textbf{VueConnexion}] Formulaire de connexion/inscription avec validation.
    
    \item[\textbf{VueJoueur, VueEquipe, VueTerrain, VueMatch, VueBut\_alle}] 
        Vues spécialisées pour la gestion des entités. Chacune étend \texttt{BaseLayout} 
        et accède aux données via \texttt{ConnectionPool}.
\end{description}

\subsection*{4. Flux d'Accès Aux Données}

\begin{enumerate}
    \item \textbf{Frontend} (Vaadin) appelle une méthode métier sur une classe \texttt{ClasseMiroir}
    \item \textbf{ClasseMiroir} obtient une connexion via \texttt{ConnectionPool.getConnection()}
    \item Exécution de la requête SQL avec filtrage par \texttt{UserSession.getCurrentTournoiId()}
    \item Résultat mappé en objets Java et renvoyé au Frontend
\end{enumerate}

\end{document}


\documentclass[12pt]{article}
\usepackage[utf-8]{inputenc}
\usepackage[french]{babel}
\usepackage{tikz}
\usepackage{tikz-uml}
\usepackage{geometry}
\geometry{landscape, margin=0.5cm}

\title{Diagramme de Classes UML - Système de Gestion de Tournois}
\author{GETP2 - Yong Ngantcha Salzenstein}
\date{\today}

\begin{document}

\begin{center}
\begin{tikzpicture}[scale=0.9, every node/.style={anchor=west}]

% ============================================================================
% BACKEND - MODEL LAYER (Métier / Persistence)
% ============================================================================

% CLASS: Utilisateur
\begin{class}[text width=4cm]{Utilisateur}{-8, 14}
    \attribute{- id: int}
    \attribute{- surnom: String}
    \attribute{- pass: String}
    \attribute{- isAdmin: boolean}
    \operation{+ getId(): int}
    \operation{+ getSurnom(): String}
    \operation{+ isAdmin(): boolean}
    \operation{+ getRole(): int}
\end{class}

% CLASS: UserSession (Singleton-like)
\begin{class}[text width=5cm]{UserSession}{-8, 8}
    \attribute{- static curUser: Utilisateur}
    \attribute{- static currentTournoiId: Integer}
    \attribute{- static currentTournoiName: String}
    \operation{+ static login(Utilisateur): void}
    \operation{+ static logout(): void}
    \operation{+ static userConnected(): boolean}
    \operation{+ static getCurrentTournoiId(): Optional<Integer>}
    \operation{+ static setCurrentTournoi(id, name): void}
\end{class}

% CLASS: Parametre (Tournament Settings)
\begin{class}[text width=4.5cm]{Parametre}{3, 14}
    \attribute{- id: int}
    \attribute{- nomTournoi: String}
    \attribute{- sport: String}
    \attribute{- nombreTerrains: int}
    \attribute{- nombreJoueursParEquipe: int}
    \operation{+ getId(): int}
    \operation{+ getNomTournoi(): String}
    \operation{+ getSport(): String}
    \operation{+ saveInDB(Connection): int}
\end{class}

% CLASS: ClasseMiroir (Abstract Base for DB Entities)
\begin{class}[text width=5cm]{«abstract» ClasseMiroir}{3, 8}
    \attribute{- id: int}
    \operation{+ getId(): int}
    \operation{+ setId(int): void}
    \operation{# abstract saveSansId(Connection): Statement}
    \operation{+ saveInDB(Connection): int}
    \operation{+ loadFromDB(Connection, int): void}
\end{class}

% CLASS: Joueur
\begin{class}[text width=4.5cm]{Joueur}{-2, 3}
    \attribute{- prenom: String}
    \attribute{- nom: String}
    \attribute{- age: int}
    \attribute{- sexe: String}
    \attribute{- taille: double}
    \operation{+ getPrenom(): String}
    \operation{+ getNom(): String}
    \operation{+ getAge(): int}
\end{class}

% CLASS: Equipe
\begin{class}[text width=4.5cm]{Equipe}{3, 3}
    \attribute{- nomEquipe: String}
    \attribute{- dateCreation: LocalDate}
    \attribute{- joueurs: List<Joueur>}
    \operation{+ getNomEquipe(): String}
    \operation{+ getJoueurs(): List<Joueur>}
    \operation{+ addJoueur(Joueur): void}
\end{class}

% CLASS: Terrain
\begin{class}[text width=4.5cm]{Terrain}{8, 3}
    \attribute{- nomTerrain: String}
    \attribute{- numero: int}
    \operation{+ getNomTerrain(): String}
    \operation{+ getNumero(): int}
    \operation{+ toString(): String}
\end{class}

% CLASS: GestionMatchs (Service Layer)
\begin{class}[text width=5.5cm]{GestionMatchs}{-8, 1}
    \attribute{- static MatchRow: record}
    \attribute{- static ButeurRow: record}
    \operation{+ static creerRencontres(): void}
    \operation{+ static calculerScores(): void}
    \operation{+ static getClassement(): List<ClassementRow>}
    \operation{+ static enregistrerBut(): void}
\end{class}

% CLASS: ConnectionPool (Database Access)
\begin{class}[text width=5cm]{ConnectionPool}{3, -2}
    \attribute{- static dataSource: HikariDataSource}
    \attribute{- static dbUrl: String}
    \operation{+ static init(): void}
    \operation{+ static getConnection(): Connection}
    \operation{+ static close(): void}
\end{class}

% ============================================================================
% FRONTEND LAYER (Vaadin Views)
% ============================================================================

% CLASS: BaseLayout (Navigation Container)
\begin{class}[text width=5cm]{BaseLayout}{-8, -4}
    \attribute{- gridCurrentTournoi: Grid}
    \attribute{- contentLayout: VerticalLayout}
    \operation{- createHeader(): void}
    \operation{- createDrawer(): void}
    \operation{- logout(): void}
    \operation{- switchTournament(): void}
\end{class}

% CLASS: VuePrincipale (Home)
\begin{class}[text width=4.5cm]{VuePrincipale}{-2, -4}
    \attribute{- container: VerticalLayout}
    \operation{+ construireInterfaceConnectee(): void}
    \operation{+ construireInterfaceNonConnectee(): void}
\end{class}

% CLASS: VueConnexion (Login/SignUp)
\begin{class}[text width=4.5cm]{VueConnexion}{3, -4}
    \attribute{- tfUsername: TextField}
    \attribute{- pfPassword: PasswordField}
    \operation{- gererConnexion(): void}
    \operation{- gererInscription(): void}
\end{class}

% CLASS: VueJoueur (Player Management)
\begin{class}[text width=4.5cm]{VueJoueur}{8, -4}
    \attribute{- grid: Grid<Joueur>}
    \attribute{- dialogForm: Dialog}
    \operation{- chargerJoueurs(): void}
    \operation{- ajouterJoueur(): void}
    \operation{- supprimerJoueur(): void}
\end{class}

% CLASS: VueEquipe (Team Management)
\begin{class}[text width=4.5cm]{VueEquipe}{-2, -8}
    \attribute{- grid: Grid<Equipe>}
    \attribute{- joueursList: List<Joueur>}
    \operation{- chargerEquipes(): void}
    \operation{- associerJoueurs(): void}
\end{class}

% CLASS: VueMatch (Match Management)
\begin{class}[text width=4.5cm]{VueMatch}{3, -8}
    \attribute{- gridRencontres: Grid}
    \attribute{- gridScores: Grid}
    \operation{- chargerMatchs(): void}
    \operation{- enregistrerScore(): void}
\end{class}

% CLASS: VueBut_alle (Scoring Board)
\begin{class}[text width=5cm]{VueBut_alle}{8, -8}
    \attribute{- gridCurrentTournoi: Grid}
    \attribute{- gridAllTournois: Grid}
    \operation{- loadButeurs(): void}
    \operation{- buildPodium(): HorizontalLayout}
\end{class}

% ============================================================================
% DATABASE LAYER (Conceptual)
% ============================================================================

% DATABASE ICON
\node[draw, rectangle, fill=cyan!20, text width=8cm, minimum height=1.5cm] at (3, -11.5) {
    \textbf{Base de Données Relationnelle} \\
    Tables: \texttt{utilisateur, parametre, joueur, equipe, terrain, rencontre, but}
};

% ============================================================================
% RELATIONSHIPS / ASSOCIATIONS
% ============================================================================

% UserSession uses Utilisateur
\draw [->, dashed, line width=1.5pt] (-2, 8) -- (1, 14);
\node[text, font=\tiny] at (0, 11) {\textbf{0..1}};

% Parametre inherits from ClasseMiroir
\draw [->, line width=1.5pt] (5.5, 8) -- (6, 12);
\node[text, font=\tiny] at (6.5, 10) {\textbf{inherits}};

% Joueur inherits from ClasseMiroir
\draw [->, line width=1.5pt] (2.5, 8) -- (0.5, 3);
\node[text, font=\tiny] at (1, 5.5) {\textbf{inherits}};

% Equipe inherits from ClasseMiroir
\draw [->, line width=1.5pt] (4.5, 8) -- (5.5, 3);
\node[text, font=\tiny] at (5.5, 5.5) {\textbf{inherits}};

% Terrain inherits from ClasseMiroir
\draw [->, line width=1.5pt] (6.5, 8) -- (10, 3);
\node[text, font=\tiny] at (8.5, 5.5) {\textbf{inherits}};

% Equipe contains Joueurs
\draw [->, line width=1.5pt] (3, 3) -- (2, 3);
\node[text, font=\tiny] at (2.5, 2.5) {\textbf{0..*}};
\node[text, font=\tiny] at (2.5, 3.3) {\textbf{contient}};

% GestionMatchs uses Joueur, Equipe, Terrain
\draw [->, dashed, line width=1pt] (-5, 1) -- (-1, 3);
\draw [->, dashed, line width=1pt] (-4, 1) -- (2, 3);
\draw [->, dashed, line width=1pt] (-3, 1) -- (8, 3);

% BaseLayout extends AppLayout (Vaadin)
\node[text, font=\tiny] at (-6, -4.5) {\textbf{extends AppLayout}};

% VuePrincipale extends BaseLayout
\draw [->, line width=1.5pt] (-2, -4) -- (-6.5, -4);
\node[text, font=\tiny] at (-4, -3.7) {\textbf{extends}};

% VueConnexion uses Utilisateur
\draw [->, dashed, line width=1pt] (4, -4) -- (0, 14);

% VueJoueur extends BaseLayout
\draw [->, line width=1.5pt] (8.5, -4) -- (-6, -4);
\node[text, font=\tiny] at (2, -3.7) {\textbf{extends}};

% VueJoueur uses Joueur
\draw [->, dashed, line width=1pt] (8.5, -4) -- (1, 3);

% VueEquipe extends BaseLayout
\draw [->, line width=1.5pt] (-2, -8) -- (-6, -4.5);
\node[text, font=\tiny] at (-4, -6) {\textbf{extends}};

% VueEquipe uses Equipe, Joueur
\draw [->, dashed, line width=1pt] (-1, -8) -- (3, 3);
\draw [->, dashed, line width=1pt] (-1.5, -8) -- (-0.5, 3);

% VueMatch extends BaseLayout
\draw [->, line width=1.5pt] (3, -8) -- (-6, -4.5);
\node[text, font=\tiny] at (-1, -6) {\textbf{extends}};

% VueMatch uses GestionMatchs
\draw [->, dashed, line width=1pt] (3, -8) -- (-5, 1);

% VueBut_alle extends BaseLayout
\draw [->, line width=1.5pt] (8.5, -8) -- (-6, -4.5);
\node[text, font=\tiny] at (2, -6) {\textbf{extends}};

% All Views access Database through ConnectionPool
\draw [->, dashed, line width=1pt, color=red] (6, -2) -- (6, -2.5);
\node[text, font=\tiny, color=red] at (6.5, -1.5) {\textbf{SQL}};

% ============================================================================
% LEGEND
% ============================================================================

\node[draw, rectangle, fill=gray!10, text width=18cm, minimum height=2cm] at (6, -13.5) {
    \textbf{Légende:} \\
    \textbf{→} Héritage / Extension \quad \textbf{⇢} Association/Utilisation \quad \textbf{- - →} Dépendance/Accès \\
    \textbf{Couches:} Backend (Model/Métier) = Classes jaunes | Frontend (Vaadin Views) = Classes bleues | DB Accès = Classes vertes
};

\end{tikzpicture}
\end{center}

\newpage

% ============================================================================
% DETAILED DESCRIPTIONS
% ============================================================================

\section*{Description Détaillée de l'Architecture}

\subsection*{1. Couche Métier (Backend Model)}

\begin{description}
    \item[\textbf{Utilisateur}] Représente un utilisateur du système avec ses droits (admin/user). 
        Stocké en base de données avec chiffrement du mot de passe.
    
    \item[\textbf{UserSession}] Gestionnaire de session singleton. Maintient l'état actuel de l'utilisateur 
        connecté et du tournoi sélectionné. Gère la transition login/logout.
    
    \item[\textbf{Parametre}] Définit les paramètres d'un tournoi (nom, sport, nombre de terrains, 
        joueurs par équipe). Hérite de \texttt{ClasseMiroir} pour la persistance.
    
    \item[\textbf{Joueur}] Entité représentant un joueur avec attributs (prénom, nom, âge, taille, sexe).
        Liée au tournoi via \texttt{tournoi\_id}.
    
    \item[\textbf{Equipe}] Groupe de joueurs avec date de création. Contient une collection de \texttt{Joueur}.
    
    \item[\textbf{Terrain}] Ressource physique pour les matchs (numéro, nom).
    
    \item[\textbf{ClasseMiroir}] Classe abstraite de base pour toutes les entités persistantes.
        Gère l'ID auto-généré et les opérations CRUD standard.
    
    \item[\textbf{GestionMatchs}] Service métier pour la création et gestion des rencontres.
        Utilise des DTOs internes (\texttt{MatchRow}, \texttt{ButeurRow}) pour les requêtes complexes.
\end{description}

\subsection*{2. Couche Données (Database Access)}

\begin{description}
    \item[\textbf{ConnectionPool}] Gestionnaire de connexions utilisant HikariCP. 
        Fournit des connexions réutilisables pour optimiser les performances.
    
    \item[\textbf{Schéma Unifié}] Une seule base de données avec colonne \texttt{tournoi\_id} 
        pour isoler les données par tournoi (pas de tables par tournoi).
\end{description}

\subsection*{3. Couche Présentation (Vaadin Frontend)}

\begin{description}
    \item[\textbf{BaseLayout}] Container racine hérité par toutes les vues. Fournit le header 
        (navigation, utilisateur, déconnexion) et le drawer (menu latéral).
    
    \item[\textbf{VuePrincipale}] Accueil avec interface différente selon l'état de connexion.
    
    \item[\textbf{VueConnexion}] Formulaire de connexion/inscription avec validation.
    
    \item[\textbf{VueJoueur, VueEquipe, VueTerrain, VueMatch, VueBut\_alle}] 
        Vues spécialisées pour la gestion des entités. Chacune étend \texttt{BaseLayout} 
        et accède aux données via \texttt{ConnectionPool}.
\end{description}

\subsection*{4. Flux d'Accès Aux Données}

\begin{enumerate}
    \item \textbf{Frontend} (Vaadin) appelle une méthode métier sur une classe \texttt{ClasseMiroir}
    \item \textbf{ClasseMiroir} obtient une connexion via \texttt{ConnectionPool.getConnection()}
    \item Exécution de la requête SQL avec filtrage par \texttt{UserSession.getCurrentTournoiId()}
    \item Résultat mappé en objets Java et renvoyé au Frontend
\end{enumerate}

\end{document}


\documentclass[12pt]{article}
\usepackage[utf-8]{inputenc}
\usepackage[french]{babel}
\usepackage{tikz}
\usepackage{tikz-uml}
\usepackage{geometry}
\geometry{landscape, margin=0.5cm}

\title{Diagramme de Classes UML - Système de Gestion de Tournois}
\author{GETP2 - Yong Ngantcha Salzenstein}
\date{\today}

\begin{document}

\begin{center}
\begin{tikzpicture}[scale=0.9, every node/.style={anchor=west}]

% ============================================================================
% BACKEND - MODEL LAYER (Métier / Persistence)
% ============================================================================

% CLASS: Utilisateur
\begin{class}[text width=4cm]{Utilisateur}{-8, 14}
    \attribute{- id: int}
    \attribute{- surnom: String}
    \attribute{- pass: String}
    \attribute{- isAdmin: boolean}
    \operation{+ getId(): int}
    \operation{+ getSurnom(): String}
    \operation{+ isAdmin(): boolean}
    \operation{+ getRole(): int}
\end{class}

% CLASS: UserSession (Singleton-like)
\begin{class}[text width=5cm]{UserSession}{-8, 8}
    \attribute{- static curUser: Utilisateur}
    \attribute{- static currentTournoiId: Integer}
    \attribute{- static currentTournoiName: String}
    \operation{+ static login(Utilisateur): void}
    \operation{+ static logout(): void}
    \operation{+ static userConnected(): boolean}
    \operation{+ static getCurrentTournoiId(): Optional<Integer>}
    \operation{+ static setCurrentTournoi(id, name): void}
\end{class}

% CLASS: Parametre (Tournament Settings)
\begin{class}[text width=4.5cm]{Parametre}{3, 14}
    \attribute{- id: int}
    \attribute{- nomTournoi: String}
    \attribute{- sport: String}
    \attribute{- nombreTerrains: int}
    \attribute{- nombreJoueursParEquipe: int}
    \operation{+ getId(): int}
    \operation{+ getNomTournoi(): String}
    \operation{+ getSport(): String}
    \operation{+ saveInDB(Connection): int}
\end{class}

% CLASS: ClasseMiroir (Abstract Base for DB Entities)
\begin{class}[text width=5cm]{«abstract» ClasseMiroir}{3, 8}
    \attribute{- id: int}
    \operation{+ getId(): int}
    \operation{+ setId(int): void}
    \operation{# abstract saveSansId(Connection): Statement}
    \operation{+ saveInDB(Connection): int}
    \operation{+ loadFromDB(Connection, int): void}
\end{class}

% CLASS: Joueur
\begin{class}[text width=4.5cm]{Joueur}{-2, 3}
    \attribute{- prenom: String}
    \attribute{- nom: String}
    \attribute{- age: int}
    \attribute{- sexe: String}
    \attribute{- taille: double}
    \operation{+ getPrenom(): String}
    \operation{+ getNom(): String}
    \operation{+ getAge(): int}
\end{class}

% CLASS: Equipe
\begin{class}[text width=4.5cm]{Equipe}{3, 3}
    \attribute{- nomEquipe: String}
    \attribute{- dateCreation: LocalDate}
    \attribute{- joueurs: List<Joueur>}
    \operation{+ getNomEquipe(): String}
    \operation{+ getJoueurs(): List<Joueur>}
    \operation{+ addJoueur(Joueur): void}
\end{class}

% CLASS: Terrain
\begin{class}[text width=4.5cm]{Terrain}{8, 3}
    \attribute{- nomTerrain: String}
    \attribute{- numero: int}
    \operation{+ getNomTerrain(): String}
    \operation{+ getNumero(): int}
    \operation{+ toString(): String}
\end{class}

% CLASS: GestionMatchs (Service Layer)
\begin{class}[text width=5.5cm]{GestionMatchs}{-8, 1}
    \attribute{- static MatchRow: record}
    \attribute{- static ButeurRow: record}
    \operation{+ static creerRencontres(): void}
    \operation{+ static calculerScores(): void}
    \operation{+ static getClassement(): List<ClassementRow>}
    \operation{+ static enregistrerBut(): void}
\end{class}

% CLASS: ConnectionPool (Database Access)
\begin{class}[text width=5cm]{ConnectionPool}{3, -2}
    \attribute{- static dataSource: HikariDataSource}
    \attribute{- static dbUrl: String}
    \operation{+ static init(): void}
    \operation{+ static getConnection(): Connection}
    \operation{+ static close(): void}
\end{class}

% ============================================================================
% FRONTEND LAYER (Vaadin Views)
% ============================================================================

% CLASS: BaseLayout (Navigation Container)
\begin{class}[text width=5cm]{BaseLayout}{-8, -4}
    \attribute{- gridCurrentTournoi: Grid}
    \attribute{- contentLayout: VerticalLayout}
    \operation{- createHeader(): void}
    \operation{- createDrawer(): void}
    \operation{- logout(): void}
    \operation{- switchTournament(): void}
\end{class}

% CLASS: VuePrincipale (Home)
\begin{class}[text width=4.5cm]{VuePrincipale}{-2, -4}
    \attribute{- container: VerticalLayout}
    \operation{+ construireInterfaceConnectee(): void}
    \operation{+ construireInterfaceNonConnectee(): void}
\end{class}

% CLASS: VueConnexion (Login/SignUp)
\begin{class}[text width=4.5cm]{VueConnexion}{3, -4}
    \attribute{- tfUsername: TextField}
    \attribute{- pfPassword: PasswordField}
    \operation{- gererConnexion(): void}
    \operation{- gererInscription(): void}
\end{class}

% CLASS: VueJoueur (Player Management)
\begin{class}[text width=4.5cm]{VueJoueur}{8, -4}
    \attribute{- grid: Grid<Joueur>}
    \attribute{- dialogForm: Dialog}
    \operation{- chargerJoueurs(): void}
    \operation{- ajouterJoueur(): void}
    \operation{- supprimerJoueur(): void}
\end{class}

% CLASS: VueEquipe (Team Management)
\begin{class}[text width=4.5cm]{VueEquipe}{-2, -8}
    \attribute{- grid: Grid<Equipe>}
    \attribute{- joueursList: List<Joueur>}
    \operation{- chargerEquipes(): void}
    \operation{- associerJoueurs(): void}
\end{class}

% CLASS: VueMatch (Match Management)
\begin{class}[text width=4.5cm]{VueMatch}{3, -8}
    \attribute{- gridRencontres: Grid}
    \attribute{- gridScores: Grid}
    \operation{- chargerMatchs(): void}
    \operation{- enregistrerScore(): void}
\end{class}

% CLASS: VueBut_alle (Scoring Board)
\begin{class}[text width=5cm]{VueBut_alle}{8, -8}
    \attribute{- gridCurrentTournoi: Grid}
    \attribute{- gridAllTournois: Grid}
    \operation{- loadButeurs(): void}
    \operation{- buildPodium(): HorizontalLayout}
\end{class}

% ============================================================================
% DATABASE LAYER (Conceptual)
% ============================================================================

% DATABASE ICON
\node[draw, rectangle, fill=cyan!20, text width=8cm, minimum height=1.5cm] at (3, -11.5) {
    \textbf{Base de Données Relationnelle} \\
    Tables: \texttt{utilisateur, parametre, joueur, equipe, terrain, rencontre, but}
};

% ============================================================================
% RELATIONSHIPS / ASSOCIATIONS
% ============================================================================

% UserSession uses Utilisateur
\draw [->, dashed, line width=1.5pt] (-2, 8) -- (1, 14);
\node[text, font=\tiny] at (0, 11) {\textbf{0..1}};

% Parametre inherits from ClasseMiroir
\draw [->, line width=1.5pt] (5.5, 8) -- (6, 12);
\node[text, font=\tiny] at (6.5, 10) {\textbf{inherits}};

% Joueur inherits from ClasseMiroir
\draw [->, line width=1.5pt] (2.5, 8) -- (0.5, 3);
\node[text, font=\tiny] at (1, 5.5) {\textbf{inherits}};

% Equipe inherits from ClasseMiroir
\draw [->, line width=1.5pt] (4.5, 8) -- (5.5, 3);
\node[text, font=\tiny] at (5.5, 5.5) {\textbf{inherits}};

% Terrain inherits from ClasseMiroir
\draw [->, line width=1.5pt] (6.5, 8) -- (10, 3);
\node[text, font=\tiny] at (8.5, 5.5) {\textbf{inherits}};

% Equipe contains Joueurs
\draw [->, line width=1.5pt] (3, 3) -- (2, 3);
\node[text, font=\tiny] at (2.5, 2.5) {\textbf{0..*}};
\node[text, font=\tiny] at (2.5, 3.3) {\textbf{contient}};

% GestionMatchs uses Joueur, Equipe, Terrain
\draw [->, dashed, line width=1pt] (-5, 1) -- (-1, 3);
\draw [->, dashed, line width=1pt] (-4, 1) -- (2, 3);
\draw [->, dashed, line width=1pt] (-3, 1) -- (8, 3);

% BaseLayout extends AppLayout (Vaadin)
\node[text, font=\tiny] at (-6, -4.5) {\textbf{extends AppLayout}};

% VuePrincipale extends BaseLayout
\draw [->, line width=1.5pt] (-2, -4) -- (-6.5, -4);
\node[text, font=\tiny] at (-4, -3.7) {\textbf{extends}};

% VueConnexion uses Utilisateur
\draw [->, dashed, line width=1pt] (4, -4) -- (0, 14);

% VueJoueur extends BaseLayout
\draw [->, line width=1.5pt] (8.5, -4) -- (-6, -4);
\node[text, font=\tiny] at (2, -3.7) {\textbf{extends}};

% VueJoueur uses Joueur
\draw [->, dashed, line width=1pt] (8.5, -4) -- (1, 3);

% VueEquipe extends BaseLayout
\draw [->, line width=1.5pt] (-2, -8) -- (-6, -4.5);
\node[text, font=\tiny] at (-4, -6) {\textbf{extends}};

% VueEquipe uses Equipe, Joueur
\draw [->, dashed, line width=1pt] (-1, -8) -- (3, 3);
\draw [->, dashed, line width=1pt] (-1.5, -8) -- (-0.5, 3);

% VueMatch extends BaseLayout
\draw [->, line width=1.5pt] (3, -8) -- (-6, -4.5);
\node[text, font=\tiny] at (-1, -6) {\textbf{extends}};

% VueMatch uses GestionMatchs
\draw [->, dashed, line width=1pt] (3, -8) -- (-5, 1);

% VueBut_alle extends BaseLayout
\draw [->, line width=1.5pt] (8.5, -8) -- (-6, -4.5);
\node[text, font=\tiny] at (2, -6) {\textbf{extends}};

% All Views access Database through ConnectionPool
\draw [->, dashed, line width=1pt, color=red] (6, -2) -- (6, -2.5);
\node[text, font=\tiny, color=red] at (6.5, -1.5) {\textbf{SQL}};

% ============================================================================
% LEGEND
% ============================================================================

\node[draw, rectangle, fill=gray!10, text width=18cm, minimum height=2cm] at (6, -13.5) {
    \textbf{Légende:} \\
    \textbf{→} Héritage / Extension \quad \textbf{⇢} Association/Utilisation \quad \textbf{- - →} Dépendance/Accès \\
    \textbf{Couches:} Backend (Model/Métier) = Classes jaunes | Frontend (Vaadin Views) = Classes bleues | DB Accès = Classes vertes
};

\end{tikzpicture}
\end{center}

\newpage

% ============================================================================
% DETAILED DESCRIPTIONS
% ============================================================================

\section*{Description Détaillée de l'Architecture}

\subsection*{1. Couche Métier (Backend Model)}

\begin{description}
    \item[\textbf{Utilisateur}] Représente un utilisateur du système avec ses droits (admin/user). 
        Stocké en base de données avec chiffrement du mot de passe.
    
    \item[\textbf{UserSession}] Gestionnaire de session singleton. Maintient l'état actuel de l'utilisateur 
        connecté et du tournoi sélectionné. Gère la transition login/logout.
    
    \item[\textbf{Parametre}] Définit les paramètres d'un tournoi (nom, sport, nombre de terrains, 
        joueurs par équipe). Hérite de \texttt{ClasseMiroir} pour la persistance.
    
    \item[\textbf{Joueur}] Entité représentant un joueur avec attributs (prénom, nom, âge, taille, sexe).
        Liée au tournoi via \texttt{tournoi\_id}.
    
    \item[\textbf{Equipe}] Groupe de joueurs avec date de création. Contient une collection de \texttt{Joueur}.
    
    \item[\textbf{Terrain}] Ressource physique pour les matchs (numéro, nom).
    
    \item[\textbf{ClasseMiroir}] Classe abstraite de base pour toutes les entités persistantes.
        Gère l'ID auto-généré et les opérations CRUD standard.
    
    \item[\textbf{GestionMatchs}] Service métier pour la création et gestion des rencontres.
        Utilise des DTOs internes (\texttt{MatchRow}, \texttt{ButeurRow}) pour les requêtes complexes.
\end{description}

\subsection*{2. Couche Données (Database Access)}

\begin{description}
    \item[\textbf{ConnectionPool}] Gestionnaire de connexions utilisant HikariCP. 
        Fournit des connexions réutilisables pour optimiser les performances.
    
    \item[\textbf{Schéma Unifié}] Une seule base de données avec colonne \texttt{tournoi\_id} 
        pour isoler les données par tournoi (pas de tables par tournoi).
\end{description}

\subsection*{3. Couche Présentation (Vaadin Frontend)}

\begin{description}
    \item[\textbf{BaseLayout}] Container racine hérité par toutes les vues. Fournit le header 
        (navigation, utilisateur, déconnexion) et le drawer (menu latéral).
    
    \item[\textbf{VuePrincipale}] Accueil avec interface différente selon l'état de connexion.
    
    \item[\textbf{VueConnexion}] Formulaire de connexion/inscription avec validation.
    
    \item[\textbf{VueJoueur, VueEquipe, VueTerrain, VueMatch, VueBut\_alle}] 
        Vues spécialisées pour la gestion des entités. Chacune étend \texttt{BaseLayout} 
        et accède aux données via \texttt{ConnectionPool}.
\end{description}

\subsection*{4. Flux d'Accès Aux Données}

\begin{enumerate}
    \item \textbf{Frontend} (Vaadin) appelle une méthode métier sur une classe \texttt{ClasseMiroir}
    \item \textbf{ClasseMiroir} obtient une connexion via \texttt{ConnectionPool.getConnection()}
    \item Exécution de la requête SQL avec filtrage par \texttt{UserSession.getCurrentTournoiId()}
    \item Résultat mappé en objets Java et renvoyé au Frontend
\end{enumerate}

\end{document}


\documentclass[12pt]{article}
\usepackage[utf-8]{inputenc}
\usepackage[french]{babel}
\usepackage{tikz}
\usepackage{tikz-uml}
\usepackage{geometry}
\geometry{landscape, margin=0.5cm}

\title{Diagramme de Classes UML - Système de Gestion de Tournois}
\author{GETP2 - Yong Ngantcha Salzenstein}
\date{\today}

\begin{document}

\begin{center}
\begin{tikzpicture}[scale=0.9, every node/.style={anchor=west}]

% ============================================================================
% BACKEND - MODEL LAYER (Métier / Persistence)
% ============================================================================

% CLASS: Utilisateur
\begin{class}[text width=4cm]{Utilisateur}{-8, 14}
    \attribute{- id: int}
    \attribute{- surnom: String}
    \attribute{- pass: String}
    \attribute{- isAdmin: boolean}
    \operation{+ getId(): int}
    \operation{+ getSurnom(): String}
    \operation{+ isAdmin(): boolean}
    \operation{+ getRole(): int}
\end{class}

% CLASS: UserSession (Singleton-like)
\begin{class}[text width=5cm]{UserSession}{-8, 8}
    \attribute{- static curUser: Utilisateur}
    \attribute{- static currentTournoiId: Integer}
    \attribute{- static currentTournoiName: String}
    \operation{+ static login(Utilisateur): void}
    \operation{+ static logout(): void}
    \operation{+ static userConnected(): boolean}
    \operation{+ static getCurrentTournoiId(): Optional<Integer>}
    \operation{+ static setCurrentTournoi(id, name): void}
\end{class}

% CLASS: Parametre (Tournament Settings)
\begin{class}[text width=4.5cm]{Parametre}{3, 14}
    \attribute{- id: int}
    \attribute{- nomTournoi: String}
    \attribute{- sport: String}
    \attribute{- nombreTerrains: int}
    \attribute{- nombreJoueursParEquipe: int}
    \operation{+ getId(): int}
    \operation{+ getNomTournoi(): String}
    \operation{+ getSport(): String}
    \operation{+ saveInDB(Connection): int}
\end{class}

% CLASS: ClasseMiroir (Abstract Base for DB Entities)
\begin{class}[text width=5cm]{«abstract» ClasseMiroir}{3, 8}
    \attribute{- id: int}
    \operation{+ getId(): int}
    \operation{+ setId(int): void}
    \operation{# abstract saveSansId(Connection): Statement}
    \operation{+ saveInDB(Connection): int}
    \operation{+ loadFromDB(Connection, int): void}
\end{class}

% CLASS: Joueur
\begin{class}[text width=4.5cm]{Joueur}{-2, 3}
    \attribute{- prenom: String}
    \attribute{- nom: String}
    \attribute{- age: int}
    \attribute{- sexe: String}
    \attribute{- taille: double}
    \operation{+ getPrenom(): String}
    \operation{+ getNom(): String}
    \operation{+ getAge(): int}
\end{class}

% CLASS: Equipe
\begin{class}[text width=4.5cm]{Equipe}{3, 3}
    \attribute{- nomEquipe: String}
    \attribute{- dateCreation: LocalDate}
    \attribute{- joueurs: List<Joueur>}
    \operation{+ getNomEquipe(): String}
    \operation{+ getJoueurs(): List<Joueur>}
    \operation{+ addJoueur(Joueur): void}
\end{class}

% CLASS: Terrain
\begin{class}[text width=4.5cm]{Terrain}{8, 3}
    \attribute{- nomTerrain: String}
    \attribute{- numero: int}
    \operation{+ getNomTerrain(): String}
    \operation{+ getNumero(): int}
    \operation{+ toString(): String}
\end{class}

% CLASS: GestionMatchs (Service Layer)
\begin{class}[text width=5.5cm]{GestionMatchs}{-8, 1}
    \attribute{- static MatchRow: record}
    \attribute{- static ButeurRow: record}
    \operation{+ static creerRencontres(): void}
    \operation{+ static calculerScores(): void}
    \operation{+ static getClassement(): List<ClassementRow>}
    \operation{+ static enregistrerBut(): void}
\end{class}

% CLASS: ConnectionPool (Database Access)
\begin{class}[text width=5cm]{ConnectionPool}{3, -2}
    \attribute{- static dataSource: HikariDataSource}
    \attribute{- static dbUrl: String}
    \operation{+ static init(): void}
    \operation{+ static getConnection(): Connection}
    \operation{+ static close(): void}
\end{class}

% ============================================================================
% FRONTEND LAYER (Vaadin Views)
% ============================================================================

% CLASS: BaseLayout (Navigation Container)
\begin{class}[text width=5cm]{BaseLayout}{-8, -4}
    \attribute{- gridCurrentTournoi: Grid}
    \attribute{- contentLayout: VerticalLayout}
    \operation{- createHeader(): void}
    \operation{- createDrawer(): void}
    \operation{- logout(): void}
    \operation{- switchTournament(): void}
\end{class}

% CLASS: VuePrincipale (Home)
\begin{class}[text width=4.5cm]{VuePrincipale}{-2, -4}
    \attribute{- container: VerticalLayout}
    \operation{+ construireInterfaceConnectee(): void}
    \operation{+ construireInterfaceNonConnectee(): void}
\end{class}

% CLASS: VueConnexion (Login/SignUp)
\begin{class}[text width=4.5cm]{VueConnexion}{3, -4}
    \attribute{- tfUsername: TextField}
    \attribute{- pfPassword: PasswordField}
    \operation{- gererConnexion(): void}
    \operation{- gererInscription(): void}
\end{class}

% CLASS: VueJoueur (Player Management)
\begin{class}[text width=4.5cm]{VueJoueur}{8, -4}
    \attribute{- grid: Grid<Joueur>}
    \attribute{- dialogForm: Dialog}
    \operation{- chargerJoueurs(): void}
    \operation{- ajouterJoueur(): void}
    \operation{- supprimerJoueur(): void}
\end{class}

% CLASS: VueEquipe (Team Management)
\begin{class}[text width=4.5cm]{VueEquipe}{-2, -8}
    \attribute{- grid: Grid<Equipe>}
    \attribute{- joueursList: List<Joueur>}
    \operation{- chargerEquipes(): void}
    \operation{- associerJoueurs(): void}
\end{class}

% CLASS: VueMatch (Match Management)
\begin{class}[text width=4.5cm]{VueMatch}{3, -8}
    \attribute{- gridRencontres: Grid}
    \attribute{- gridScores: Grid}
    \operation{- chargerMatchs(): void}
    \operation{- enregistrerScore(): void}
\end{class}

% CLASS: VueBut_alle (Scoring Board)
\begin{class}[text width=5cm]{VueBut_alle}{8, -8}
    \attribute{- gridCurrentTournoi: Grid}
    \attribute{- gridAllTournois: Grid}
    \operation{- loadButeurs(): void}
    \operation{- buildPodium(): HorizontalLayout}
\end{class}

% ============================================================================
% DATABASE LAYER (Conceptual)
% ============================================================================

% DATABASE ICON
\node[draw, rectangle, fill=cyan!20, text width=8cm, minimum height=1.5cm] at (3, -11.5) {
    \textbf{Base de Données Relationnelle} \\
    Tables: \texttt{utilisateur, parametre, joueur, equipe, terrain, rencontre, but}
};

% ============================================================================
% RELATIONSHIPS / ASSOCIATIONS
% ============================================================================

% UserSession uses Utilisateur
\draw [->, dashed, line width=1.5pt] (-2, 8) -- (1, 14);
\node[text, font=\tiny] at (0, 11) {\textbf{0..1}};

% Parametre inherits from ClasseMiroir
\draw [->, line width=1.5pt] (5.5, 8) -- (6, 12);
\node[text, font=\tiny] at (6.5, 10) {\textbf{inherits}};

% Joueur inherits from ClasseMiroir
\draw [->, line width=1.5pt] (2.5, 8) -- (0.5, 3);
\node[text, font=\tiny] at (1, 5.5) {\textbf{inherits}};

% Equipe inherits from ClasseMiroir
\draw [->, line width=1.5pt] (4.5, 8) -- (5.5, 3);
\node[text, font=\tiny] at (5.5, 5.5) {\textbf{inherits}};

% Terrain inherits from ClasseMiroir
\draw [->, line width=1.5pt] (6.5, 8) -- (10, 3);
\node[text, font=\tiny] at (8.5, 5.5) {\textbf{inherits}};

% Equipe contains Joueurs
\draw [->, line width=1.5pt] (3, 3) -- (2, 3);
\node[text, font=\tiny] at (2.5, 2.5) {\textbf{0..*}};
\node[text, font=\tiny] at (2.5, 3.3) {\textbf{contient}};

% GestionMatchs uses Joueur, Equipe, Terrain
\draw [->, dashed, line width=1pt] (-5, 1) -- (-1, 3);
\draw [->, dashed, line width=1pt] (-4, 1) -- (2, 3);
\draw [->, dashed, line width=1pt] (-3, 1) -- (8, 3);

% BaseLayout extends AppLayout (Vaadin)
\node[text, font=\tiny] at (-6, -4.5) {\textbf{extends AppLayout}};

% VuePrincipale extends BaseLayout
\draw [->, line width=1.5pt] (-2, -4) -- (-6.5, -4);
\node[text, font=\tiny] at (-4, -3.7) {\textbf{extends}};

% VueConnexion uses Utilisateur
\draw [->, dashed, line width=1pt] (4, -4) -- (0, 14);

% VueJoueur extends BaseLayout
\draw [->, line width=1.5pt] (8.5, -4) -- (-6, -4);
\node[text, font=\tiny] at (2, -3.7) {\textbf{extends}};

% VueJoueur uses Joueur
\draw [->, dashed, line width=1pt] (8.5, -4) -- (1, 3);

% VueEquipe extends BaseLayout
\draw [->, line width=1.5pt] (-2, -8) -- (-6, -4.5);
\node[text, font=\tiny] at (-4, -6) {\textbf{extends}};

% VueEquipe uses Equipe, Joueur
\draw [->, dashed, line width=1pt] (-1, -8) -- (3, 3);
\draw [->, dashed, line width=1pt] (-1.5, -8) -- (-0.5, 3);

% VueMatch extends BaseLayout
\draw [->, line width=1.5pt] (3, -8) -- (-6, -4.5);
\node[text, font=\tiny] at (-1, -6) {\textbf{extends}};

% VueMatch uses GestionMatchs
\draw [->, dashed, line width=1pt] (3, -8) -- (-5, 1);

% VueBut_alle extends BaseLayout
\draw [->, line width=1.5pt] (8.5, -8) -- (-6, -4.5);
\node[text, font=\tiny] at (2, -6) {\textbf{extends}};

% All Views access Database through ConnectionPool
\draw [->, dashed, line width=1pt, color=red] (6, -2) -- (6, -2.5);
\node[text, font=\tiny, color=red] at (6.5, -1.5) {\textbf{SQL}};

% ============================================================================
% LEGEND
% ============================================================================

\node[draw, rectangle, fill=gray!10, text width=18cm, minimum height=2cm] at (6, -13.5) {
    \textbf{Légende:} \\
    \textbf{→} Héritage / Extension \quad \textbf{⇢} Association/Utilisation \quad \textbf{- - →} Dépendance/Accès \\
    \textbf{Couches:} Backend (Model/Métier) = Classes jaunes | Frontend (Vaadin Views) = Classes bleues | DB Accès = Classes vertes
};

\end{tikzpicture}
\end{center}

\newpage

% ============================================================================
% DETAILED DESCRIPTIONS
% ============================================================================

\section*{Description Détaillée de l'Architecture}

\subsection*{1. Couche Métier (Backend Model)}

\begin{description}
    \item[\textbf{Utilisateur}] Représente un utilisateur du système avec ses droits (admin/user). 
        Stocké en base de données avec chiffrement du mot de passe.
    
    \item[\textbf{UserSession}] Gestionnaire de session singleton. Maintient l'état actuel de l'utilisateur 
        connecté et du tournoi sélectionné. Gère la transition login/logout.
    
    \item[\textbf{Parametre}] Définit les paramètres d'un tournoi (nom, sport, nombre de terrains, 
        joueurs par équipe). Hérite de \texttt{ClasseMiroir} pour la persistance.
    
    \item[\textbf{Joueur}] Entité représentant un joueur avec attributs (prénom, nom, âge, taille, sexe).
        Liée au tournoi via \texttt{tournoi\_id}.
    
    \item[\textbf{Equipe}] Groupe de joueurs avec date de création. Contient une collection de \texttt{Joueur}.
    
    \item[\textbf{Terrain}] Ressource physique pour les matchs (numéro, nom).
    
    \item[\textbf{ClasseMiroir}] Classe abstraite de base pour toutes les entités persistantes.
        Gère l'ID auto-généré et les opérations CRUD standard.
    
    \item[\textbf{GestionMatchs}] Service métier pour la création et gestion des rencontres.
        Utilise des DTOs internes (\texttt{MatchRow}, \texttt{ButeurRow}) pour les requêtes complexes.
\end{description}

\subsection*{2. Couche Données (Database Access)}

\begin{description}
    \item[\textbf{ConnectionPool}] Gestionnaire de connexions utilisant HikariCP. 
        Fournit des connexions réutilisables pour optimiser les performances.
    
    \item[\textbf{Schéma Unifié}] Une seule base de données avec colonne \texttt{tournoi\_id} 
        pour isoler les données par tournoi (pas de tables par tournoi).
\end{description}

\subsection*{3. Couche Présentation (Vaadin Frontend)}

\begin{description}
    \item[\textbf{BaseLayout}] Container racine hérité par toutes les vues. Fournit le header 
        (navigation, utilisateur, déconnexion) et le drawer (menu latéral).
    
    \item[\textbf{VuePrincipale}] Accueil avec interface différente selon l'état de connexion.
    
    \item[\textbf{VueConnexion}] Formulaire de connexion/inscription avec validation.
    
    \item[\textbf{VueJoueur, VueEquipe, VueTerrain, VueMatch, VueBut\_alle}] 
        Vues spécialisées pour la gestion des entités. Chacune étend \texttt{BaseLayout} 
        et accède aux données via \texttt{ConnectionPool}.
\end{description}

\subsection*{4. Flux d'Accès Aux Données}

\begin{enumerate}
    \item \textbf{Frontend} (Vaadin) appelle une méthode métier sur une classe \texttt{ClasseMiroir}
    \item \textbf{ClasseMiroir} obtient une connexion via \texttt{ConnectionPool.getConnection()}
    \item Exécution de la requête SQL avec filtrage par \texttt{UserSession.getCurrentTournoiId()}
    \item Résultat mappé en objets Java et renvoyé au Frontend
\end{enumerate}

\end{document}
